「お疲れ様でした!」本書を読み終えられた人に、まずはそう申し上げたい。同時に、構想から1年以上をかけ、ようやくテキストを書き終えた自分にも同じ言葉を贈りたい。
\\ \\ 
近年、「C言語は古びた言語であるから、学ぶ価値がない」というような論調を見聞する。たしかに、C++やJavaなどのより強力な言語があるし、LLも数多く出ている現在、Cを実践で使う機会は減ってきているのかも知れない。だが、私は「C言語を一度学ぶことはどの言語を主で使うにしても有用である」と思う。これは例えば、英文学を考えてみればわかるだろう。シェイクスピアの作品は古典と言われる部類に属するだろうが、非常に豊かな語彙をはじめとして、現代から見ても決して読む価値のないものではない。それどころか、シェイクスピアを知らないことは無教養とさえされているところがある。我が国で言えば、明治から大正にかけての文学は、たしかに古いものであるが、そこから学ぶところは決して少なくないだろう。
\\ \\ 
この様な重要な古典としてC言語を捉えた時、多くの言語に影響を与えていることがわかる。C言語を学ぶことは、現代よく用いられる言語の背景にある考えを理解することである。科学を見ても歴史を見ても、背景の考えを理解することが有用なのは言うまでもないことであろう。
\\ \\ 
これらの「背景の考え」を初心者のうちから理解していれば、他の言語の学習も捗る。この考えから、本書は「C言語の文法を中心に、背景の考えをよく理解できる説明」を行うようにしたつもりである。それがどの程度達成されているかは読者諸賢の判断を待つしかないが…。
\\ \\ 
本書は、理論の面からプログラミングの楽しさを伝えようとしてみたものであるが、果たして上手く伝えることができただろうか。その答えは、読者の皆様がこれからプログラミングに携わっていく中で自ずと見えてくることだろう。もしも本書を読むことによってプログラミング好きが一人でも増えたのであれば、本書の目的は達成されたといえるだろう。
\\ \\ 
本テキストを利用してくれた人たち、本テキストの執筆に携わってくれた全ての人たちに心よりの感謝を捧げ、筆を置くことにする。
\begin{flushright}
中島みゆき「ローリング」を聞きながら\\
達哉ん
\end{flushright}

