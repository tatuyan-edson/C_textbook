構造化定理において挙げられる3構造のうち、分岐構造について学ぶ。
\section{分岐構造とは}
実際に使い方に入る前に、分岐構造とは何なのか、説明しておこう。
\\ \\ 
\textbf{分岐構造}\index{ぶんきこうぞう@分岐構造}というのは、何らかの条件を与え、それによって処理を変えるという構造である。例えば、RPGにおいて、「はい」「いいえ」の選択肢がある時、選択肢に応じて会話が変わる。平方根を計算するプログラムを組む時、もし、与えられる数が負の数だったらエラーを出力する。このように「条件の真偽に応じて行う処理を変える」のが分岐構造の役目である。

\section{if文とelse文}
\subsection{if文}
まず、最も単純な分岐を考えてみよう。
\begin{boxnote}
\begin{multicols}{2}
\minisec{0割回避機能付き四則演算}
二つの整数の四則演算を行うプログラムを作成する。この時、割り算については、除数が0でない場合のみ実行するようにする。
\minisec{解説}
四則演算そのものは前回までに習ったものと同じである。これに、もしも2つ目の数が0でないならば割り算を計算して出力するという命令を書き加える。
\begin{lstlisting}[caption=0割回避四則演算,label=program4_1]
#include<stdio.h>

int main(void){
  int a,b;
  scanf("%d%d",&a,&b);
  printf("Sum :%d\n",a+b);
  printf("Dif :%d\n",a-b);
  printf("Prod:%d\n",a*b);
  if(b!=0){
    printf("Div :%d\n",a/b);
  }
  return 0;
}
\end{lstlisting}
\end{multicols}
\end{boxnote}
\minisec{if文の使い方}
今回目新しいのは、$l$.9のif文であろう。これは、次のような形式で用いられる。
\begin{itembox}[l]{if文}
if文は
\begin{code}
if(条件式){
  処理
}
\end{code}
の形式で記し、条件式が真の場合に中括弧内の処理が行われる。
\\ \\
なお、処理が一文の場合
\begin{code}
if(条件式) 処理;
\end{code}
の形でも書ける。(但し、書き直しの際に中括弧を入れる必要が出てくる場合も多いので、多用はされない。)

\end{itembox}

if文を用いることにより、単純な分岐を作ることができる。また、if文の中に更にif文を入れることもできる。このようにあるブロック内に更に別のブロックを入れることを\textbf{ネスト}\index{ねすと@ネスト}(nest)あるいは\textbf{入れ子}\index{いれこ@入れ子|see{ネスト}}と呼ぶ。後に習う各ブロック(if,else,switch,for,while,doなど)は任意に組み合わせてネスト構造にすることができる。

\minisec{条件式とその値}
if文の引数は\textbf{条件式}である。この条件式は、表\ref{relational_ope}に示す\textbf{比較演算子}\index{ひかくえんざんし@比較演算子}あるいは\textbf{関係演算子}\index{かんけいえんざんし@関係演算子|see{比較演算子}}(relational operator)を用いて書くことができる。
\begin{table}[htb]
\centering
\caption{比較演算子の一覧}\label{relational_ope}
\begin{tabular}{|c|c|}\hline
記号&意味\\ \hline
& \\[-15.5pt] \hline
\verb|==| & 左辺と右辺が等しい\\ \hline
\verb|!=| & 左辺と右辺が等しくない\\ \hline
\verb|<| & 左辺が右辺より小さい\\ \hline
\verb|<=| & 左辺が右辺以下である\\ \hline
\verb|>| & 左辺が右辺より大きい\\ \hline
\verb|>=| & 左辺が右辺以上である\\ \hline
\end{tabular}
\end{table}

これらによって比較が行われた時、条件式が真であれば1、偽であれば0と評価され、その値が利用できる。例えば、\verb|((a==1)+(b==-1))==2|などとすると、これは、\verb|a|が1で\verb|b|が-1の時だけ真になる条件式になる。この条件式の値を利用すれば、ここまでの知識だけでも様々な条件を書くことができる。

逆に、一般の式の値を真/偽の値としても利用することができる。一般の式の値について、0は偽を示し、それ以外の値(非0)は真を示す。例えば、
\begin{code}
if(n%2){
  処理
}
\end{code}
とすると、この処理は\verb|n|が奇数の場合のみに処理が行われることになる。一般に、条件式のあるべきところに単なる値が書かれているときには、後ろに\verb|!=0|を補えば良い。このことから、\verb|if(1)|は必ず実行されるし、\verb|if(0)|は実行されないとわかる。
\\ \\ 
条件式の値はif文だけにとどまらない重要事項であるので、以下にまとめておく。
\begin{itembox}[l]{条件式と値}
\begin{itemize}
\item 条件式は真の時に1と、偽の時に0と評価される。
\item 一般の式が条件式として用いられた場合、その値が非0の場合は真として、0の場合は偽として扱われる。
\end{itemize}
\end{itembox}
\\ \\ 
最後に、比較演算子の注意点を記しておく。
\begin{itembox}[l]{比較演算子の注意点}
\begin{itemize}
\item 比較演算子は二項演算であるので、左辺と右辺の比較しかできない。つまり、\verb|a==b==c|や\verb|a<=b<c|のような書き方は思ったのと違う評価になる\footnote{このような論理式を実現するには、後述の論理演算子を用いるか、if文のネストを組めば良い。}。
\item 等しいことを示す\verb|==|は必ず二つ書かねばならない。これを\verb|=|とするバグは非常に多く、またコンパイルエラーにならないことが多いため、見つけづらい\footnote{これを回避する方法として、変数==値と書いているものを、逆に値==変数と書けば良いと言われることもある。だが、ソースが見づらくなる割に、特定の場合にしか効果を発揮せず、推奨しがたい。似非イディオムと言われることさえあるので、極力使わないほうが良い。}。
\item \verb|<=|と\verb|>=|、\verb|!=|は\verb|=|を後ろ側に書かなければならない。「小なりイコール」などと読む順番に書くと覚えれば良い。
\item 比較演算子はビットシフトより遅く評価されるが、他のビット演算とは同等で、左側から評価される。
\end{itemize}
\end{itembox}


\subsection{else文}
ここまでは単なる「○○の時に〜する」形の分岐だけを扱ったが、「そうでなければ〜〜する」という分岐を加えることができる。例えば、「リストに登録されたメールアドレスであれば受信し、それ以外であれば遮断する」の「それ以外」に当たる部分である。
\\ \\ 
ここでは、偶数か奇数かを判別して出力するプログラムを考えてみよう。
\begin{boxnote}
\begin{multicols}{2}
\minisec{偶奇の判定}
入力される自然数が偶数か奇数かを判定するプログラムを作成する。
\minisec{解説}
単純な条件式は\verb|a%2|の形であるが、2による剰余は2進法の一番下の位が0か1かを判定することで実装できる。そのため、ここでは\verb|a&1|という形で実装してみた(条件式の値を思い出そう)。
\begin{lstlisting}[caption=偶奇の判定,label=program4_2]
#include<stdio.h>

int main(void){
  unsigned int a;
  scanf("%u",&a);
  if(a&1){
    puts("Odd number.");
  }else{
    puts("Even number.");
  }
  return 0;
}
\end{lstlisting}
\end{multicols}
\end{boxnote}
\minisec{else文}
ここで新しく出てきたelse文こそが、「そうでなければ〜〜する」を実行するための文である。
\begin{itembox}[l]{else文}
else文は、前述のifに続けて
\begin{code}
if(条件式){
  処理
}else{
  処理
}
\end{code}
の形で書き、対応するif節の条件式が偽である場合のみに実行される。なお、対応するif節とは、同じ階層(ネストの深さが同じ)であるifのうち、直前のifを指す。例えば、
\begin{code}
if(...){ //if-1
  if(...) a=b; //if-2
  else b=a; //if-2に対応
}else(...){ //if-1に対応
 ...
}
\end{code}
といった具合である。

\end{itembox}
\\ \\ 
このelse文を用いると「Aならば○○、Bならば〜〜する」と書けるが、これにとどまらず、else文は更にその後ろにif節を取ることができる。
\begin{itembox}[l]{else if文}
elseの直後にifを記し、
\begin{code}
if(条件式1){
  処理1
}else if(条件式2){
  処理2
  :
}else if(条件式n){
  処理n
}else{
  処理else
}
\end{code}
のような形で書けば、「条件式k-1までに合致せず条件式kに合致する」場合に、処理kが行われる。
\end{itembox}
これにより、多岐分岐を実装することができる。多岐分岐には後述のswitch〜caseが使われることもあるが、こちらのほうが活用される範囲が広いので、有効活用されたい。
\\ \\ 
なお、else文はif文に対するオプションのようなものであり、ifに対して必ず対応するelseを書かなければならないというわけではない。したがって、else ifを連続した末尾にelse節が必ず来るとは限らない。逆に、elseには必ずそれに対応するif節が必要である。

\section{論理演算子}
ここまでで単一の条件による判定と、条件式による判定を学んできた。だが、$a,b,c$が全て1であると表現するのに、\verb|(a==1)+(b==1)+(c==1)==3|のような式で書くのは見づらい。このような複数条件の条件式に用いるのが\textbf{論理演算子}\index{ろんりえんざんし@論理演算子}(Logical operator)である。
\begin{boxnote}
\minisec{閏年の判定}
入力される年が閏年かどうかを判定するプログラムを作成する。
\minisec{解説}
グレゴリオ暦で考える場合、閏年は4で割りきれて100で割り切れないか、あるいは400で割り切れる年である。
\end{boxnote}
\begin{boxnote}
\begin{lstlisting}[caption=閏年の判定,label=program4_3]
#include<stdio.h>

int main(void){
  unsigned int y;
  scanf("%u",&y);
  if(y%4==0 && y%100!=0 || !(y%400)){
    puts("Leap year");
  }else{
    puts("Not Leap year");
  }
  return 0;
}
\end{lstlisting}
\end{boxnote}
\minisec{論理演算子の意味}
なんと言っても条件式に出てくる、\verb|&&|、\verb\||\,\verb|!|であろう。これは、表\ref{logical_ope}のような意味を持つ。これらによって条件式を複数組み合わせて書くことができる(\textbf{複合条件文}\index{ふくごうじょうけんぶん@複合条件文})。
\begin{table}[htb]
\begin{center}
\caption{論理演算子}\label{logical_ope}
\begin{tabular}{|c|c|}\hline
記号&意味\\ \hline
& \\[-15.5pt] \hline
\verb|&&|&かつ(連言)\\ \hline
\verb`||` &または(選言)\\ \hline
\verb|!| &でない(否定)\\ \hline
\end{tabular}
\end{center}
\end{table}

これらの演算子はビット演算と似ているが、ビット演算はビットごとに演算するのに対し、論理演算は論理値のみに対して演算を行う点が違う。論理値は先に述べたとおり、非0を真、0を偽として扱い、逆に条件式から論理値が値として出る場合は真が1、偽が0となる。
\begin{itembox}[l]{論理演算子とその値}
\begin{itemize}
\item \verb|(条件式A)&&(条件式B)|は、条件式A,Bとも真(非0)である場合には真(1)を返し、それ以外は偽(0)を返す。
\item \verb'(条件式A)||(条件式B)'は、条件式A,Bの一方が真(非0)である場合には真(1)を返し、両方が偽(0)の場合は偽(0)を返す。
\item \verb|!(条件式)|は、条件式が真(非0)の場合には偽(0)を返し、偽(0)の場合には真(1)を返す。
\end{itemize}
\end{itembox}

\minisec{論理演算子の計算順序と評価}
論理演算子の評価順序および評価の仕組みについて説明する。
\begin{itembox}[l]{論理演算子の評価順序}
\begin{itemize}
\item \verb|!|演算子は単項演算子として扱われ、評価順序も他の単項演算子と同じである。
\item \verb|&&|演算子は比較演算、ビット演算の後に評価される。但し、結合前後の論理値によって評価に影響を与える。
\item \verb'||'演算子は、\verb|&&|演算の後に評価される。但し、結合前後の論理値によって評価に影響を与える。
\item どの論理演算子も、左側にあるものを先に評価する。
\end{itemize}
\end{itembox}

この説明のうち、気になるのが「結合前後の論理値によって評価に影響を与える。」という文面であろう。これについて、\verb|&&|を例にとって説明する。

\verb|&&|演算子について、例えば、\verb|(a==5)&&(b%2)|という式があったとしよう。これは左側から展開されるので、まず、\verb|(a==5)|が評価される。そして、仮にこれが偽の場合、\verb|&&|演算子の値はその段階で定まってしまうので、以降の\verb|(b%2)|が評価されない。
\\ \\ 
このように、論理演算子は、その複合条件文の値が定まった場合、その段階で評価を打ち切ってしまう。この性質は、有効利用すればプログラムの高速化につながる反面、例えば\verb|(a==2)&&(b++)|などと式を書いた場合、\verb|a|が2でなければ\verb|b|のインクリメントが行われないために意図した結果と違う結果をもたらす場合もある\footnote{一方で、コードの短さを競う競技である"Code Golf"では、この性質を用いた書き換えを行うのが定石になっている。これは、\verb|if(a==0) b=1;|などの文について、\verb'a||b=1;'などとして、文字数をカットできるためである。勿論この書き方は、"Code Golf"競技以外では通常使わない。}。評価順序は原則的に上に記したとおりだが、論理演算を行う場合はこの打ち切り処理に注意しなければならない。慣れてくるまではこの処理を有効利用して高速化するなどは考えず(実際、体感できるほど速くなることは、初心者のうちはないと言って良い)、後に影響を残す可能性のある演算(代入演算、インクリメント/デクリメント)を複合論理式中に組み込まないように注意を払ったほうが良い。

なお、論理式が1または0で評価されることを考えると、ビット演算子でも似たようなことができると気づくだろう。だが、ビット演算には、ここで書いた値決定時の評価打ち切りがないという違いがある。XOR等を使えるなどの利点もあるので、使うべきではないとは言わないが、使う際には注意されたい。

\section{switch〜case文}
ここまで、条件式を用いた条件判定を行って来たが、C言語にはもうひとつ、値の列挙による条件判定機能がある。値の列挙とは、現実世界でいうところの「唐揚げ弁当ならば500円、トンカツ弁当ならば550円、エビフライ弁当は580円...」のような、多数の選択肢から1つを選ぶ場合どうするか、というものである。C言語の場合は、この値の列挙には「それ以外」が加わり、例えば「定食のご飯が白飯なら500円、玄米なら520円、五穀米なら550円、それ以外なら580円」といった形の選択になる。この実装を見ていこう。
\begin{boxnote}
\begin{multicols}{2}
\minisec{計算式を打ち込む四則演算}
整数と演算子(\verb|+,-,*,/|)からなる式を打ち込んでもらい、その値を計算するプログラムを作成する。
\minisec{解説}
\begin{itemize}
\item 演算子の判別が必要になる。演算子は範囲などでは書けないので、ここではじめて、先に書いた「値の列挙」、ここでは文字の列挙による判定が行われる。
\item データの入力は、整数,文字,整数なので、scanfの書式文字列を\verb|"%d%c%d"|としてやれば良い。
\item 途中のelse節($l$.20-22)の中にある\verb|return 0;|はmain関数の終了を示し、仮にこのelseのところの処理が行われることになったら、ここでmain関数の処理が打ち切られるということを示す。このように、main関数の途中に\verb|return|をおいてmain関数を終了させる手法はよく用いられる。(特に、0以外の値を指定して、異常終了という形にすることが多い。)
\end{itemize}
\begin{lstlisting}[caption=計算式を打ち込む四則演算,label=program4_4]
#include<stdio.h>

int main(void){
  int a,b,ans,flg=0;
  char c;
  scanf("%d%c%d",&a,&c,&b);
  switch(c){
    case '+':
      ans=a+b;
      break;
    case '-':
      ans=a-b;
      break;
    case '*':
      ans=a*b;
      break;
    case '/':
      if(b!=0){
        ans=a/b;
      }else{
        return 0;
      }
      break;
    default:
      flg=1;
      break;
  }
  if(flg){
    puts("error");
  }else{
    printf("%d\n",ans);
  }
  return 0;
}
\end{lstlisting}
\end{multicols}
\end{boxnote}

\minisec{switch caseによる分岐}
今回用いた値の分岐は、switch〜case文による。このswitch〜case文は先に書いたとおり、定数値による分岐を行う。公式の前に、流れを解説しておく。

switch部分に来ると、switchの引数の式を評価し、それと同じ定数式をブロック中のcaseから探し求める。見つかったならば、該当するcaseまで飛び、見つからなければdefaultまで飛んで、その後の処理を実行する\footnote{実は、このcaseないしdefaultというのは、ラベルという文である。ラベルは、\verb|ラベル名:|という形で書き、次講で解説するgoto文(構造化プログラミングの提唱論文において、流れを見難くするので乱用すべきでないとされた命令)の引数にとることができる。switch内のラベルはcaseを用いるかdefaultを用いるかしかないが、これにgotoを使ってアクセスすると流れが見づらい\textbf{スパゲッティソースコード}\index{すぱげってぃそーすこーど@スパゲッティソースコード}があっという間にでき上がる。これは可読性を著しく下げるので、switch中に飛ぶようなgotoを作ってはならない。この理由のため、一部の書籍では、switchそのものを用いるべきでないとするものまである。}。途中で\verb|break|という命令が見つかった場合は、その段階でswitchを抜ける。なお、このcase文を多数書くとコードが長くなるが、一応256個までであればコンパイルを通ることが保証されている\footnote{256個というのは、文字コードに応じた文字を返す処理を作るときに必要であるからである。通常、そんなに沢山のcaseを使うことはない。なお、処理系によっては256個より多くのcase文が許される場合もある。}。
\\ \\ 
通常、switch〜case文はbreakを用いて、次のように書く。
\begin{itembox}[l]{switch〜case文}
switch〜case文は
\begin{code}
switch(判定値){
  case 定数式:
    処理
    break;
     :
  default:
    処理
    break;
}
\end{code}
の形で書く。
\end{itembox}
switch〜caseにおいてdefaultは省略してもよい。省略してcaseの中で該当する値が見つからなかった場合は何も処理が実行されないことになる。また、同様にdefaultの終端の\verb|break|も省略することができるが、ラベルを書き換えたときに忘れがちなので、書くようにしている人も多い。

なお、switch〜caseもネストすることはできるが、ひどく見づらいため、めったに使わない。switch〜caseの中で更に条件分岐が必要な場合は、ifを使うほうが良いだろう。

\minisec{breakなしのswitch〜case}
続いて、\verb|break|を使わないswitch〜caseについて見ていこう。

\verb|break|を書かないと、その後の処理が引き続き行われる。例えば、
\begin{code}
switch(a){
  case 1:
    puts("a=1");
  case 2:
    puts("a=2");
  default:
    puts("a>=3 or a<=0");
}
\end{code}
というコードがあった時、もしも\verb|a|が1だったならば、出力は
\begin{code}
a=1
a=2
a>=3 or a<=0
\end{code}
となる。
\\ \\ 
これは、\verb|break|を忘れた場合に処理がおかしくなるということであるが、逆に利用することもできる。最後に、その例を紹介しておく。
\begin{boxnote}
\begin{multicols}{2}
\minisec{月の日数}
月が入力される時、その月の日数を出力する。
\minisec{解説}
ここでは、breakなしのcase文を一部(4月、6月、9月の場合の処理)に用いて、処理を簡単化した。breakなしの場合、そのまま下にうつって\verb|30days.|と出力される。\\ \\なお、閏年の処理は施しておらず、1から12以外の数が入れられた時の処理も行っていない。これらについては、switchの前後にif文を設ければ、比較的簡単に処理ができる。
\begin{lstlisting}[caption=月の日数,label=program4_5]
#include<stdio.h>

int main(void){
  short month;
  scanf("%hd",&month);
  switch(month){
    case 4:
    case 6:
    case 9:
    case 11:
      puts("30days.");
      break;
    case 2:
      puts("28days.");
      break;
    default:
      puts("31days.");
      break;
  }
  return 0;
}
\end{lstlisting}
\end{multicols}
\end{boxnote}

なお、breakなしswitch構造では、defaultラベルがswitch文の末尾にこないこともある。

\section{条件演算子}
「ある条件の時にはこの計算式を用い、そうでない場合にはこの計算式を用いる」のような、ごく単純な(計算だけの)場合分けをいちいちif文,else文で書いていると非常に面倒である。そこで、式中で使えるような分岐を作る演算子が用意されている。この分岐を行う演算子が\textbf{条件演算子}\index{じょうけんえんざんし@条件演算子}(conditional operator)である。この演算子は、C言語で唯一の三項演算子であり、C言語中ではしばしば三項演算子と条件演算子が同じ意味で用いられる。では、二数の比較を例に、場合分けを見てみよう。
\begin{boxnote}
\begin{multicols}{2}
\minisec{二数の比較}
入力される二つの整数のうち、大きい側の値を出力する。
\minisec{解説}
if文を用いる代わりに、条件演算子を用いて作成してみる。なお、条件演算子は、後に学ぶマクロと相性がよく、主にマクロの定義に用いられる。
\begin{lstlisting}[caption=二数の比較,label=program4_6]
#include<stdio.h>

int main(void){
  int m,n,max;
  scanf("%d%d",&m,&n);
  max=(m>n)?m:n;
  printf("%d\n",max);
  return 0;
}
\end{lstlisting}
\end{multicols}
\end{boxnote}

このように、条件演算子は、\verb|?|と\verb|:|からなり以下の形式で用いる。
\begin{itembox}[l]{条件演算子の形式}
\begin{code}
(条件式)?(真の場合の式):(偽の場合の式)
\end{code}
\end{itembox}
この時、条件式の値に応じて真の場合/偽の場合のいずれか一方が評価される。この、式を評価する性質のため、リスト\ref{program4_6}のように、代入の右辺値等として使われる。

\section{\_Bool型の利用}
C99では条件式の値を保存するための型として、\_Bool型が新たに作られた。これは、条件値の保持や返却に有効である。以下に、その特徴を示す。
\begin{itembox}[l]{\_Bool型の特徴}
\begin{itemize}
\item \_Bool型は符号なし型として扱われ、signed, unsignedはつけられない。
\item \_Bool型は0と1を格納できれば十分なサイズとして定義されている。
\item \_Bool型に型変換する際は、元の値が0なら0、非0なら1に変換される。
\item \_Bool型は最もサイズの小さい整数型として扱われ、他の型との演算の際には優先度が一番低い。
\item \_Bool型は整数型であるのでビット演算を行うことができる。とりわけ、XORを表現するのに利便性が高い。
\end{itemize}
\end{itembox}
\minisec{\_Bool型を扱うヘッダ}
この\_Bool型を利用するために、C99においてstdbool.h\footnote{STanDard BOOLから。}というヘッダが追加された。実際に\_Bool型を利用する際には、このヘッダを使う場合が多いだろう。以下、利用上の手引きを示す。
\begin{itembox}[l]{stdbool.hの利用方法}
\begin{itemize}
\item このヘッダはC99から導入されたヘッダであるので、対応していないコンパイラでは利用できない。
\item このヘッダをインクルードした場合、型名を\_Boolと書かずboolと書いて良いようになる。(complex.hをインクルードした時のcomplexやimaginaryと同じことである)
\item このヘッダをインクルードした場合、\verb|true|/\verb|false|を用いて真値(1)/偽値(0)を表現することができるようになる。プログラムを書く際に、条件式の値をあてにする場合、この\verb|true|/\verb|false|を利用するとソースが読み易くなる。
\item (参考)このヘッダには、\verb|__bool_true_false_are_defined|というマクロが定義されている。このマクロを利用することにより、後に解説する列挙型などを用いて\_Bool型と同等の機能を定義した場合の判別ができるようになっている。
\end{itemize}
\end{itembox}

これらとビット演算子ないし論理演算子を組み合わせることで、論理代数(ブール代数)の計算を行うことができる。条件式の値に関する法則と共に有効活用すると良い。
\newpage

\begin{shadebox}
\section*{本講の要点}
ここまで、条件分岐について学んできた。以下、解説の流れとは別でまとめた。
\subsection*{条件式}
\begin{itemize}
\item 条件式は比較演算子や論理演算子を用いて書くことができる。
\item 論理演算子は式の値が決まった段階で評価を打ち切る。
\item 条件式は真値として1を、偽値として0を取る。これらのための型として\_Bool型が、それに対応したヘッダとしてstdbool.hが用意されている。
\item 一般の式を条件として用いた場合、非0を真値、0を偽値として扱う。
\end{itemize}

\subsection*{様々な分岐}
\begin{itemize}
\item 条件分岐をするには、if〜else,switch〜case,条件演算子などを用いる。
\item if〜else文は条件分岐に以下の形式で用いられ、最も多用される分岐である。
\begin{verbatim}
if(条件式){
  処理
}else{
 処理
}
\end{verbatim}
\item switch〜case文は定数値を列挙する形での分岐に用いられ、次の形式で記す。
\begin{verbatim}
switch(値){
case 定数式:
  処理
    break;
 :
default:
  処理
    break;
}
\end{verbatim}
\item switch節中でのbreakは、直上のswitch節からの抜け出しを意味する。
\item 条件演算子は、条件に応じて変化する式を書く際などの値を返すだけの分岐に用い、次の形式で記す:\verb|(条件式)?(真の場合の値):(偽の場合の値)|
\item ここにあげた分岐(及び次節で学ぶ反復)はいずれもネストすることができる。
\end{itemize}
\end{shadebox}
