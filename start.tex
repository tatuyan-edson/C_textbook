プログラミングは楽しい。その思いを伝えたいと思って、後輩にC言語を教えてきた。意欲ある後輩がプログラミングを嫌いにならぬよう、実力がつくよう、微力ながら手助けができればと思って、多くのプリントを作り、解説をしてきた。それをひと通り集めたテキストを書こうと考えて執筆したのがこのテキストである。\\

実は、このテキストの執筆を始める2年ほど前から、別のテキストを執筆して、教える際に使っていた。そのテキストは高校生を対象にしたものであるが、無料という点を除いてさしたる成功を収めたとは言いがたいものであった。自身がテキストを用いて解説すると不満がよくわかる。それらの不満を解消し、また、様々な制約を外し、自身が教えるスタイルをそのままテキストにしよう、と心がけた。\\

理科系の大学生を対象にして書いたものであるが、これは具体的には以下のような知識・技能・環境を前提としている。
\begin{itemize}
\item 初等関数の微分積分・数列や級数の極限・ベクトルや行列の計算ができる。
\item パソコンのタイピングができ、オフィス・メール・ブラウザ・テキストエディタ等を利用できる。
\item ファイルパスや拡張子の意味を理解しており、CUIを苦にしない。
\item 中学生レベルの英語の読み書きができる。
\item インターネット環境がある。
\end{itemize}
このうち、インターネット環境以外については多少の学習で身につけられるものであるので、本書を読まれる際に迷った場合があれば適当な書籍等で復習されたい。理系の方でなくとも、上記を知っていれば、あるいは並行して学習すれば本書を糧と出来るだろう。\\

筆者がC言語を教える際には、文法にとどまらず、プログラミング全体のセンスが身につくように努めている。それを本書でも活かすため、アルゴリズムなどの回を取り入れた。逆に演習は全て別に任せ(この理論編と合わせて、演習を中心として学ぶ「実習編」も編みたいと考えているが)、本書においては全15回でプログラミング全体に対する基本が身につくように徹底したつもりである。なお、自習または補習の目的として第0講を配し、学習の前に知っておくと良いと考えられる知識と、学習に必要な環境を構築する方法を述べるようにした。\\

プログラミングは一つのツールであるが、同時に一つの趣味をも提供しうるものである。この実例はまさしく筆者であり、一方で物理の理解のためのシミュレーションにプログラミングを用い、他方で競技プログラミングに参加している。これはプログラミングの工芸的側面を示している。つまり、役立つものを作るという工学的観点と、美しいものを作るという芸術的観点である。創作活動たるプログラミングは、パズルをとくものから自由気ままな創作に至るまで幅広く、それは単に美しいだけではなくて実利的側面を兼ね備え得る。本書はこの見地にたって、ツールとして実用に耐えうるうものでありながらそれを趣味として楽しめるように解説を記すよう心がけた。実用としてプログラミングを学ぶ必要もあろうが、その創作的楽しみを十分味わってもらえるように切に願っている。そして、本書を読んだ後に、実用的にプログラミングを使いつつ競技プログラミングを楽しむ方がいらっしゃったなら幸甚である。\\

なお、先に述べたとおり、本書は思い切って演習を省き、演習については姉妹書「実習編」にまとめた。同時利用を前提としての執筆であり、先に記した筆者のスタンスはこれら2冊によって為される。共にご活用されることを切に願う。\\

また、本書のタイトルについてであるが、これは私が指導に当たった方からの「この講座は理論を損なうことなく、実践も丁寧にやっている講座だ」という声を基にしたものである。このスタンスを本書でも貫き、巷間の専門書とはまた一味違ったものになっていれば、楽しく読んでいただけるのではないか。その思いから本書のタイトルを「理論と実習の両面から学ぶ」とした。\\

最後になったが、前テキストの共同執筆者の方々、筆者の拙い解説を聴きに来てくれた後輩たち、本テキストを利用してくださる方々に感謝の意を表して、本書の序としたい。
\begin{flushright}
中島みゆき「誕生」のかかる部屋にて\\
達哉ん
\end{flushright}

公私ともさまざまにあり、姉妹書「実習編」の執筆作業もあまり進んでいない中ではあるが、公開より1年余を経た2013年9月、誤植の修正をはじめ、内容のまずい部分(生硬な部分・内容が不十分である部分)の修正、一部内容の追記を行った。

幸い、自身の状況が落ち着いてきたので、この修正版公開を皮切りに実習編の執筆作業に本腰を入れていきたいと考えている。
\begin{flushright}
中島みゆき「月はそこにいる」を口ずさみながら\\
達哉ん
\end{flushright}
