本書は全15講の解説と、付録からなっている。また、前提知識をざっと学ぶ第0講も用意している。

第1講から第12講まではC言語の文法を学ぶ。通読し、姉妹書「実習編」を用いて演習をこなすことで、C言語およびプログラミングの基礎が身につくように構成したつもりである。

第13講以降は、アルゴリズムなどについての基礎を学ぶ。これを通じて文法の復習をすると共に、実装できる幅を広げることができるように構成した。
\\ \\ 
基本的には解説を通読されることをお勧めしたい。その時、各講の学習の後に姉妹書「実習編」で実際に手を動かすことにより復習されることをすすめる。十分な演習の重要性は今ここで再強調するまでもないことだろう。継続的に、復習も含めた演習を行うことでプログラミング力の底上げができることは疑いない。十分な活用を願っている。\\ 

通読の際、ソースがあればそれは必ず手を動かして打ち込み、実行することをすすめる。実際の動作なくしてプログラミングを習得することはできない。また、動作確認そのものがプログラミングの理論の理解にもつながることだろう。それ以外は、普通に読んでもらって構わない。まとめながらでも良いし、線を引きながらでも良い。\\ 

筆者がプログラミング初学者にすすめている学び方を紹介しておく。
\begin{itemize}
\item 基本的な文法事項、関数、アルゴリズムなどはノートにごく簡単にまとめておき、辞書のように使えるようにする。
\item ソースに出会ったら、必ずそれを手ずから打ち込んで、動作を確認する。
\item プログラミング中は紅茶などを用意し、ゆっくり落ち着いて、時間と心に余裕をもって考えること。
\item パソコンばかりでなく、紙とペンも用意し、内容を整理しながらプログラミングすること。
\item 多くの問題に挑戦し、自由に作品を作り、出来ればプログラミングの出来る人に批評を求めること。
\end{itemize}

これらを守ってプログラミングの学習をされることで、飛躍的に効果が出ることだろう。これからの学習が順調に進むことを祈っている。
