本書を読む上で、あるいは読まれた後で参考となる書籍・サイトを紹介しておく。なお、参考となる書籍/サイトはこの他にも多数あるが、ここでは目についたものを紹介するにとどめた。実際には、学習者自身の目で本・サイトを選ばれたい。

\subsection*{文法の解説}
\begin{itemize}
 \item C言語プログラミング(H.M.ダイテル/P.J.ダイテル著 小嶋隆一訳・ピアソンエデュケーション):現在出ているCの文法書籍の中で、最も練習問題が多いもののひとつ。説明もなかなかわかり易く、内容も浅くない。
 \item C実践プログラミング(S.オウアルライン著  望月康司監訳・オライリージャパン):上記ダイテルに比べてやや難しめであるが、内容に妥協のないCの文法書。
 \item 美しいCプログラミング見本帖(柏原正三著・翔泳社):C言語の難所「ポインタ」の観点から、「習うより慣れろ」の精神で文法を見ていく好著。ポインタのみを書いている書籍よりも広い視点でCを学ぶことができる。
 \item 本当は怖いC言語(種田元樹著・秀和システム):本書同様、C言語を理論的に説明した好著。エラーを出すプログラムを掲載し、それにより学ぶなど随所に工夫が見られる。C99にも対応している。
 \item 苦しんで覚えるC言語(\url{http://9cguide.appspot.com/}):私の知る限り最も丁寧な解説サイト。Windowsでの環境構築から全て書いてあり、非常にわかりやすい説明が行われている。最近書籍版が出た。
 \item 目指せプログラマー(\url{http://www5c.biglobe.ne.jp/~ecb/index.html}|):C言語を始め幾つかの言語と、アルゴリズム論の基礎を解説しているサイト。
 \item Programming Place Plus(\url{http://www.geocities.jp/ky_webid/index.html}):他に見られない項目も含めて細かくCを解説しているサイト。
\end{itemize}

\subsection*{Cリファレンス}
\begin{itemize}
 \item Cリファレンスマニュアル 第5版(S.P.ハービソン/G.L.スティールJr.著 玉井浩訳・SIBアクセス):現在出ているCの辞書では最も充実しているもののひとつ。
 \item C言語によるプログラミングスーパーリファレンス編(内田智史他著・オーム社):初心者向けのCの辞書だが、1999年のCの改定に対応していないのが難点。
\end{itemize}

\subsection*{その他、参考になる本}
\begin{itemize}
 \item プログラミング言語C第2版(カーニハン,リッチー著 石田晴久訳・共立出版):C言語の原典である書籍。
 \item エキスパートCプログラミング(P.リンデン著 梅原系訳・アスキー):C言語の内容についてかなり深く突っ込んだ書籍。
 \item C/C++の「迷信」と「誤解」(高木信尚著・技術評論社):非常に細かい、勘違いしやすい場所について仕様を基に突っ込んでいく書籍。
 \item 補講C言語(平田豊著・工学社):入門書では余り触れられない実践テクニックを紹介する書籍。
 \item C言語によるはじめてのアルゴリズム入門(河西朝雄著・技術評論社):アルゴリズムについて、C言語で、簡単なものを中心に紹介した好著。
 \item C言語による最新アルゴリズム事典(奥村晴彦著・技術評論社):「最新」ではなくなってきたが、C言語を用いて多くのアルゴリズムが書かれている。
 \item アルゴリズム演習300題(橋本英美著・日刊工業新聞社):簡単な問題を中心としたプログラミングの演習書。初歩的な問題も掲載されており、練習に調度良い。
 \item 演習でマスターするC言語とデータ構造(内藤広志,斉藤隆著・共立出版):データ構造と派生型について1冊通して演習する書籍。
 \item プログラマのための論理パズル(D.E.シャシャ著 吉平健治訳・オーム社):プログラミングに必要な様々な発想をパズルを元に学ぶ書籍。アルゴリズムの能力を上げるのに役立つ。
 \item Short Coding(Ozy著・毎日コミュニケーションズ):C言語で「できる限り短いソースを書く」ことに視点を当てて書かれた書籍。この書籍を楽しむことでC言語やアルゴリズムへの深い理解を得ることができる。お遊びの本といえばそうなのであるが、驚くほど有用である。
 \item プログラミングコンテストチャレンジブック(秋葉拓哉他著・毎日コミュニケーションズ):競技プログラミングに焦点を当てたプログラミングの書籍。
 \item ニューメリカル・レシピ・イン・シー(W.H.プレス他著 奥村晴彦他訳・技術評論社):数値計算に関する、標準的な参考書。
 \item ロベールのC++入門講座(ロベール著・毎日コミュニケーションズ):C++の入門書であるが、前半5章はCにも共通する部分であり、非常に良い解説が行われている。引き続きC++を学ぶ気があるならば推奨する。ページ数の割に安い。
 \item ストラウストラップのプログラミング入門(B.ストラウストラップ著 遠藤美代子訳・翔泳社):C++を作った人が書いた、プログラミング全般をC++を用いて入門するという内容の書。
\end{itemize}
