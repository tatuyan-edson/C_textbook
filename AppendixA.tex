C言語の簡易リファレンスを掲載しておく。紙面の都合上、記述は最小限度にとどめたので、利用方法についてはmanコマンドはじめインターネット等で調べて頂きたい。なお、以下の書籍・サイトを参考にした。
\begin{itemize}
\item C言語によるプログラミング[応用編]第2版(内田他著、Ohm社刊)
\item C言語によるプログラミング[スーパーリファレンス編](内田他著、Ohm社刊)
\item C言語プログラミング(H.M.Deitel他著、ピアソン・エデュケーション刊)
\item プログラミング言語C 第2版 (B.W.Kernighan他著、共立出版刊)
\item Cリファレンスマニュアル第5版(S.P.Harbison他著、SiB access刊)(C99)
\item プログラミング言語Cの新機能(\verb|http://seclan.dll.jp/c99d/|)(C99)
\item C言語関数辞典(\verb|http://www.c-tipsref.com/|)(C99)
\end{itemize}
なお、末尾にC99と付したものはC95,C99で追加された機能について用いたものである。\\

特に断らない限り、大文字で書いているものはマクロであり、型・引数を書いているものは関数である。なお、ヘッダ内で返却値の型や引数の取り方が同じである関数については冒頭にそれを断った上で関数名のみを記している(ctype.hやmath.hなど)。また、便宜上、C89の部分とC95/C99の部分は分けて書き、C95/C99で追加された関数やヘッダはC89の後に追加した。

\section{assert.h(プログラム診断)}
\begin{itemize}
\item \verb|NDEBUG|:定義することで\verb|assert()|を無効にする。
\item \verb|void assert(int expression)|:実行時の条件チェック。\verb|expression|が偽の際にファイル名と行番号を\verb|stderr|に出力して\verb|abort|する。
\end{itemize}
 
\section{ctype.h(文字の分類)}
本ヘッダで定義される関数はいずれも\verb|int function(int argument)|の形である。以下、関数名のみ記している。また、主語「引数が」を省略している。
\begin{itemize}
\begin{multicols}{2}
\item \verb|isalnum|:半角英数字ならば真。
\item \verb|isalpha|:アルファベットならば真。
\item \verb|islower|:英小文字ならば真。
\item \verb|isupper|:英大文字ならば真。
\item \verb|isdigit|:数字ならば真。
\item \verb|isxdigit|:16進数ならば真。
\item \verb|iscntrl|:制御文字ならば真。
\item \verb|ispunct|:区切り文字ならば真。
\item \verb|isspace|:空白文字ならば真。
\end{multicols}
\item \verb|isgraph|:スペース以外の印字可能文字ならば真。
\item \verb|isprint|:スペースを含む印字可能文字ならば真。
\item \verb|tolower|:英大文字であるならば対応する英小文字を返し、それ以外の場合は引数の値を返す。
\item \verb|toupper|:英小文字であるならば対応する英大文字を返し、それ以外の場合は引数の値を返す。
\end{itemize}
以下は、C99において追加された関数である。
\begin{itemize}
\item \verb|isblank|:行中空白ならば真。
\end{itemize}

\section{errno.h(エラー)}
\begin{itemize}
\begin{multicols}{2}
\item \verb|EDOM|:定義域エラー
\item \verb|ERANGE|:値域エラー
\end{multicols}
\item \verb|errno|:エラー状態を保持する外部変数
\end{itemize}

\section{float.h(浮動小数点数型属性検査)}
いずれも環境依存のマクロである。
\begin{itemize}
\item \verb|DBL_DIG|:\verb|double|型変数が10進数で表すことのできる精度の桁数
\item \verb|DBL_EPSILON|:\verb|double|型変数でのマシンイプシロンの値
\item \verb|DBL_MANT_DIG|:\verb|double|型変数の仮数部における基数\verb|FLT_RADIX|の桁数
\item \verb|DBL_MAX|:\verb|double|型変数の表現可能な最大値
\item \verb|DBL_MAX_10_EXP|:\verb|double|型変数の表現可能な指数最大値(基数10)
\item \verb|DBL_MAX_EXP|:\verb|double|型変数の表現可能な指数最大値(基数2)
\item \verb|DBL_MIN|:\verb|double|型変数の表現可能な正の最小値
\item \verb|DBL_MIN_10_EXP|:\verb|double|型変数の表現可能な指数最小値(基数10)
\item \verb|DBL_MIN_EXP|:\verb|double|型変数の表現可能な指数最小値(基数2)
\item \verb|DBL_ROUNDS|:\verb|double|型変数の足し算に対する丸めモード
\item \verb|FLT_DIG|:\verb|float|型変数が10進数で表すことのできる精度の桁数
\item \verb|FLT_EPSILON|:\verb|float|型変数でのマシンイプシロンの値
\item \verb|FLT_MANT_DIG|:\verb|float|型変数の仮数部における基数\verb|FLT_RADIX|の桁数
\item \verb|FLT_MAX|:\verb|float|型変数の表現可能な最大値
\item \verb|FLT_MAX_10_EXP|:\verb|float|型変数の表現可能な指数最大値(基数10)
\item \verb|FLT_MAX_EXP|:\verb|float|型変数の表現可能な指数最大値(基数2)
\item \verb|FLT_MIN|:\verb|float|型変数の表現可能な正の最小値
\item \verb|FLT_MIN_10_EXP|:\verb|float|型変数の表現可能な指数最小値(基数10)
\item \verb|FLT_MIN_EXP|:\verb|float|型変数の表現可能な指数最小値(基数2)
\item \verb|FLT_RADIX|:\verb|float|型変数の指数表現の基数
\item \verb|FLT_ROUNDS|:\verb|float|型変数の足し算に対する丸めモード
\item \verb|LDBL_DIG|:\verb|long double|型変数が10進数で表すことのできる精度の桁数
\item \verb|LDBL_EPSILON|:\verb|long double|型変数でのマシンイプシロンの値
\item \verb|LDBL_MANT_DIG|:\verb|long double|型変数の仮数部における基数\verb|FLT_RADIX|の桁数
\item \verb|LDBL_MAX|:\verb|long double|型変数の表現可能な最大値
\item \verb|LDBL_MAX_10_EXP|:\verb|long double|型変数の表現可能な指数最大値(基数10)
\item \verb|LDBL_MAX_EXP|:\verb|long double|型変数の表現可能な指数最大値(基数2)
\item \verb|LDBL_MIN|:\verb|long double|型変数の表現可能な正の最小値
\item \verb|LDBL_MIN_10_EXP|:\verb|long double|型変数の表現可能な指数最小値(基数10)
\item \verb|LDBL_MIN_EXP|:\verb|long double|型変数の表現可能な指数最小値(基数2)
\item \verb|LDBL_ROUNDS|:\verb|long double|型変数の足し算に対する丸めモード
\end{itemize}
以下はC99において追加されたマクロである。
\begin{itemize}
\item \verb|DECIMAL_DIG|:|:浮動小数点数型で表現できる最大の10進桁数
\item \verb|FLT_EVAL_METHOD|:実際に浮動小数点数演算を行うときの範囲と精度を示す値
\end{itemize}

\section{limits.h(整数型属性検査)}
いずれも環境依存のマクロである。
\begin{itemize}
\item \verb|CHAR_BIT|:\verb|char|型変数のビット数
\item \verb|CHAR_MAX|:\verb|char|型変数の表現可能な最大値
\item \verb|CHAR_MIN|:\verb|char|型変数の表現可能な最小値
\item \verb|INT_MAX|:\verb|int|型変数の表現可能な最大値
\item \verb|INT_MIN|:\verb|int|型変数の表現可能な最小値
\item \verb|LONG_MAX|:\verb|long|型変数の表現可能な最大値
\item \verb|LONG_MIN|:\verb|long|型変数の表現可能な最小値
\item \verb|MB_LEN_MAX|:多バイト文字の最大バイト数
\item \verb|SCHAR_MAX|:\verb|signed char|型変数の表現可能な最大値
\item \verb|SCHAR_MIN|:\verb|signed char|型変数の表現可能な最小値
\item \verb|SHRT_MAX|:\verb|short|型変数の表現可能な最大値
\item \verb|SHRT_MIN|:\verb|short|型変数の表現可能な最小値
\item \verb|UCHAR_MAX|:\verb|unsigned char|型変数の表現可能な最大値
\item \verb|UINT_MAX|:\verb|unsigned int|型変数の表現可能な最大値
\item \verb|ULONG_MAX|:\verb|unsigned long|型変数の表現可能な最大値
\item \verb|USHRT_MAX|:\verb|unsigned short|型変数の表現可能な最大値
\end{itemize}
以下は、C99において追加されたマクロである。
\begin{itemize}
\item \verb|LLONG_MAX|:\verb|long long|型変数の表現可能な最大値
\item \verb|LLONG_MIN|:\verb|long long|型変数の表現可能な最小値
\item \verb|ULLONG_MAX|:\verb|unsigned long long|型変数の表現可能な最大値
\end{itemize}

\section{locale.h(地域情報管理)}
\begin{itemize}
\item \verb|LC_ALL|:全ての地域情報に対する検索・設定用定数
\item \verb|LC_COLLATE|:地域固有の文字の比較順序情報に対する検索・設定用定数
\item \verb|LC_CTYPE|:地域固有文字(多バイト文字等)に対する検索・設定用定数
\item \verb|LC_MONETARY|:地域固有の通貨文字に対する検索・設定用定数
\item \verb|LC_NUMERIC|:地域固有の小数点文字に対する検索・設定用定数
\item \verb|LC_TIME|:地域固有の時間表現文字列に対する検索・設定用定数
\item \verb|NULL|:\verb|NULL|ポインタ
\item \verb|struct lconv|:地域固有の表現情報を格納する構造体
\item \verb|char *setlocale(int category,const char *locale)|:地域情報を設定する。\\ \verb|locale|が\verb|NULL|の時は地域情報を検索する。
\item \verb|struct lconv *localeconv(void)|:地域情報が格納された構造体\verb|lconv|へのポインタを返す。
\end{itemize}

\section{math.h(数学関数)}
gccでコンパイルする際には\verb|-lm|オプションをつけ\verb|gcc source.c -lm|とする必要がある。

以下特に断らない限り、型が省略されたものを\verb|double|型とする。
\begin{itemize}
\item \verb|HUGE_VAL|:\verb|double|型変数の表現可能な最大値(パソコンにおける「十分大きい値」)。C99ではこれにF,Lをつけたものもある。
\begin{multicols}{2}
\item \verb|acos(x)|:逆余弦を返す。$\cos^{-1}(x)$
\item \verb|asin(x)|:逆正弦を返す。$\sin^{-1}(x)$
\item \verb|atan(x)|:逆正接を返す。$\tan^{-1}(x)$
\item \verb|atan2(y,x)|:点$(x,y)$の方向角を返す。$\tan^{-1}(y/x)$。
\item \verb|cos(x)|:余弦を返す。$\cos x$
\item \verb|sin(x)|:正弦を返す。$\sin x$
\item \verb|tan(x)|:正接を返す。$\tan x$
\item \verb|cosh(x)|:双曲余弦を返す。$\cosh x$
\item \verb|sinh(x)|:双曲正弦を返す。$\sinh x$
\item \verb|tanh(x)|:双曲正接を返す。$\tanh x$
\end{multicols}
\item \verb|exp(x)|:引数に対するネイピア数$e=2.718281828\cdots$を底とする指数関数の値を返す。$e^x=\exp x$
\item \verb|frexp(value,int *exp)|:\verb|value|を$x\times 2^t$の形に表し(但し$0.5\le x<1$)、$t$を\verb|*exp|に格納した後、$x$を返す。
\item \verb|ldexp(x,int exp)|:$x\times 2^{\verb|exp|}$を返す。
\item \verb|log(x)|:自然対数を返す。$\ln x=\log_e x$
\item \verb|log10(x)|:常用対数を返す。$\log_{10} x$
\item \verb|modf(value,double* iptr)|:\verb|value|の整数部を\verb|*iptr|に格納したあと、\verb|value|の小数部の値を返す。
\item \verb|pow(x,y)|:$x^y=\exp_x y$の値を返す。
\item \verb|sqrt(x)|:正平方根値$\sqrt{x}$を返す。
\item \verb|ceil(x)|:天井関数値(小数点以下切り上げ値)を返す。$\lceil x \rceil=[x]+1$
\item \verb|floor(x)|:床関数値(小数点以下切り捨て値)を返す。$\lfloor x \rfloor=[x]$
\item \verb|fabs(x)|:絶対値を返す。$|x|$
\item \verb|fmod(x,y)|:$x$の$y$による剰余を返す。
\end{itemize}
C99では大幅に関数・マクロ等が増えた。以下、C99で追加された分である。
\begin{itemize}
\item \verb|INFINITY|:正または符号無しの無限大を示す定数マクロ
\item \verb|NAN|:浮動小数点数型のNANを示す定数マクロ
\item \verb|FP_INFINITE|:正または負の無限大を表すマクロ
\item \verb|FP_NAN|:NANを示すマクロ
\item \verb|FP_NORMAL|:正規化数(正常に表される浮動小数点数)を示すマクロ
\item \verb|FP_SUBNORMAL|:非正規化数(値が小さすぎて正常に表されない浮動小数点数)を示すマクロ
\item \verb|FP_ZERO|:0を表すマクロ
\item \verb|FP_FAST_FMA|:\verb|fma|関数が有意であることを示すマクロで、F,Lをつけたものもある(それぞれ\verb|fmaf,fmal|に対応)
\item \verb|FP_ILOGB0|:\verb|ilogb(0)|の返却値を示すマクロ
\item \verb|FP_ILOGBNAN|:\verb|ilogb(NAN)|の返却値を示すマクロ
\item \verb|MATH_ERRNO|:整数定数1に展開されるマクロ
\item \verb|MATH_ERREXCEPT|:整数定数2に展開されるマクロ
\item \verb|math_errhandling|:\verb|MATH_ERRNO|、\verb|MATH_ERREXCEPT|もしくはこの二つのビット毎論理和に展開されるマクロ
\item \verb|fpclassify(x)|:引数の値をカテゴリに分類する関数マクロ
\item \verb|isfinite(x)|:引数の値が有限の値かどうかを判定する関数マクロ
\item \verb|isinf(x)|:引数の値が無限大かどうかを判定する関数マクロ
\item \verb|isnan|:引数の値が NaN (非数) かどうかを判定する関数マクロ
\item \verb|isnormal(x)|:引数の値が正規化数かどうかを判定する関数マクロ
\item \verb|signbit(x)|:引数の符号が負かどうかを判定する関数マクロ
\item \verb|isgreater(x,y)|:\verb|x|が\verb|y|より大きいかどうかを判定する関数マクロ
\item \verb|isgreaterequal(x,y)|:\verb|x|が\verb|y|より大きい,または等しいかどうかを判定する関数マクロ
\item \verb|isless(x,y)|:\verb|x|が\verb|y|より小さいかどうかを判定する関数マクロ
\item \verb|islessequal(x,y)|:\verb|x|が\verb|y|より小さい,または等しいかどうかを判定する関数マクロ
\item \verb|islessgreater(x,y)|:\verb|x|が\verb|y|より小さい,または大きいかどうかを判定する関数マクロ
\item \verb|isunordered(x,y)|:引数が順序付け可能かどうかを判定する関数マクロ
\item \verb|acosh(x)|:逆双曲線余弦値を返す。$\cosh^{-1}(x)$
\item \verb|asinh(x)|:逆双曲線正弦値を返す。$\sinh^{-1}(x)$
\item \verb|atanh(x)|:逆双曲線正接値を返す。$\tanh^{-1}(x)$
\item \verb|exp2(x)|:引数に対する2を底とする指数関数の値を返す。$2^x=\exp_2 x$
\item \verb|expm1(x)|引数に対して$e^x-1$の値を返す。
\item \verb|log2(x)|:二進対数を返す。$\lg x=\log_2 x$
\item \verb|log1p(x)|:引数に対して$\ln (x+1)$を返す。
\item \verb|logb(x)|:引数の浮動小数点数表現における指数部分を浮動小数点数形式の符号付き整数として返す。
\item \verb|int ilogb(x)|:引数の浮動小数点数表現における指数部分を浮動小数点数形式の符号付き整数として返す。
\item \verb|scalbn(x,int n)|:$x\times$\verb|FLT_RADIX|$^n$を効率よく計算する。\verb|FLT_RADIX|については\\ \verb|float.h|参照。\verb|scalbln|という名前に変えたら、\verb|n|が\verb|long|型になる。
\item \verb|cbrt(x)|:3乗根を返す。$\sqrt[3]{x}$
\item \verb|hypot(x,y)|:過度のオーバーフローやアンダーフローを避けて入力点の原点からの距離を計算する。$\sqrt{x^2+y^2}$
\item \verb|erf(x)|:誤差関数値を返す。$\frac{2}{\sqrt{\pi}}\int^{x}_{0} \exp (-t^2) dt$
\item \verb|erfc(x)|:余誤差関数値を返す。$\frac{2}{\sqrt{\pi}}\int^{\infty}_{x} \exp (-t^2) dt$
\item \verb|tgamma(x)|:ガンマ関数値を返す。$\Gamma(x)=\int^{\infty}_{0} t^{x-1}e^{-t}dt$
\item \verb|lgamma(x)|:引数に対して$\ln|\Gamma(x)|$を返す。
\item \verb|nearbyint(x)|:引数を現在の丸め方向にしたがって浮動小数点数形式の整数値に丸める。例外は発生しない。
\item \verb|rint(x)|:引数を現在の丸め方向にしたがって浮動小数点数形式の整数値に丸める。例外が発生することがある。
\item \verb|long lrint(x)|:引数を現在の丸め方向にしたがって\verb|long|型整数値に丸める。\verb|llrint|とすれば、\verb|long long|型に丸める。
\item \verb|round(x)|:引数を四捨五入する。
\item \verb|long lround(x)|:引数を四捨五入して\verb|long|型にする。\verb|llround|とすれば、\verb|long long|型にする。
\item \verb|trunc(x)|:絶対値の小数点以下を切り捨てて返す。
\item \verb|remainder(x,y)|:IEEE60559規格に則った剰余を返す。\verb|remquo|も同様。
\item \verb|copysign(x,y)|:$x$の絶対値と$y$の符号を持った値を返す。$\frac{y}{|y|}|x|$
\item \verb|nan(const char *p)|:文字列をNaNに変換する。
\item \verb|nextafter(x,y)|:$y$方向に見たとき、$x$の次に表現可能な数値を返却する。\verb|nexttoword|関数はこの第2引数が\verb|long double|になったもの。
\item \verb|fmax(x,y)|:二つの引数のうち大きい方の値を返す。
\item \verb|fmin(x,y)|:二つの引数のうち小さい方の値を返す。
\item \verb|fdim(x,y)|:$x>y$なら$x-y$を、そうでなければ0を返す。
\item \verb|fma(x,y,z)|:$x\times y+z$を返却する。
\end{itemize}
また、ここには記さないが、C99では上記に挙げた関数の関数名終端に\verb|f|を付すと関数・引数とも\verb|float|型に、\verb|l|を付すと\verb|long double|型になる(\verb|double|以外で特記しているものを除く)。C99では\verb|math.h|も\verb|complex.h|も含め、\verb|tgmath.h|をインクルードすることで\verb|math.h|の\verb|double|型関数名で関数が使えるようになる。

\section{setjmp.h(非局所分岐)}
\begin{itemize}
\item \verb|jmp_buf|:\verb|longjmp|時に必要な環境を格納するための配列型
\item \verb|void longjmp(jmp_buf env,int val)|:\verb|env|に格納した環境を復元して\verb|setjmp|の位置に戻る(第2引数は\verb|setjmp|の返却値)。
\item \verb|int setjmp(jmp_buf env)|:\verb|longjmp|時のための環境を\verb|env|に格納する。
\end{itemize}

\section{signal.h(シグナル処理)}
\begin{itemize}
\item \verb|sig_atomic_t|:シグナルを保持する変数の型
\item \verb|SIG_DFL|:シグナルに対して処理系のデフォルトの処理を指示するマクロ
\item \verb|SIG_ERR|:シグナルの設定失敗を示すマクロ
\item \verb|SIG_IGN|:シグナルを無視することを示すマクロ
\begin{multicols}{2}
\item \verb|SIGABRT|:異常終了
\item \verb|SIGFPE|:算術演算エラー
\item \verb|SIGILL|:不正命令
\item \verb|SIGINT|:非同期割り込み
\item \verb|SIGSEGV|:メモリの不正アクセス
\item \verb|SIGTERM|:プログラム終了
\end{multicols}
\item \verb|(*signal(int sig,void(*func)(int)))(int)|:シグナル(\verb|sig|)が発生した際の処理関数設定。返却値は成功時その関数へのポインタ、失敗時\verb|SIG_ERR|。
\item \verb|int raise(int sig)|:シグナル(\verb|sig|)を発生させる。成功時は0,失敗時は非0を返す。
\end{itemize}

\section{stdarg.h(可変引数)}
\begin{itemize}
\item \verb|va_list|:可変引数リストを扱う変数の型
\item \verb|type va_arg(va_list ap,type)|:可変引数リスト\verb|ap|より\verb|type|で示す引数を返す。
\item \verb|void va_end(va_list ap)|:可変引数リスト\verb|ap|を処理して終了させる。
\item \verb|void va_start(va_list ap,parmN)|:\verb|parmN|の次から始まる可変引数リスト\verb|ap|の初期化を行う。
\end{itemize}
以下はC99で追加された関数である。
\begin{itemize}
\item \verb|void va_copy(va_list dst, va_list src)|:\verb|va_start|で\verb|va_list|を初期化したオブジェクト\verb|src|のコピー\verb|dst|を作成する。
\end{itemize}

\section{stddef.h(共通定義)}
\begin{itemize}
\item \verb|NULL|:\verb|NULL|ポインタ
\item \verb|ptrdiff_t|:2つのポインタの差を表現する型
\item \verb|size_t|:\verb|sizeof|演算子の返す符号なし整数型
\item \verb|wchar_t|:\verb|wide|文字の型
\item \verb|offsetof(type,member-designator)|:構造体\verb|type|内での構造体メンバ\\ \verb|member-designator|のオフセットを返す。
\end{itemize}

\section{stdio.h(標準入出力)}
以下、特に断らない限り、\verb|stream|は\verb|FILE *stream|を示し、ストリームポインタとする。
\begin{itemize}
\item \verb|FILE|:ファイルストリーム格納変数型
\item \verb|fpos_t|:ファイル内の位置を表す型
\item \verb|size_t|:\verb|sizeof|演算子の返す符号なし整数型
\item \verb|_IOFBF|:フルバッファリングでストリーム入出力を行う(\verb|setvbuf|の\verb|mode|)
\item \verb|_IOLBF|:ラインバッファリングでストリーム入出力を行う(\verb|setvbuf|の\verb|mode|)
\item \verb|_IONBF|:バッファリングせずにストリーム入出力を行う(\verb|setvbuf|の\verb|mode|)
\item \verb|BUFSIZ|:\verb|setbuf|で使用するバッファサイズ
\item \verb|EOF|:ファイル終端
\item \verb|FILENAME_MAX|:ファイル名の最大長
\item \verb|FOPEN_MAX|:同時オープン可能ファイル個数
\item \verb|L_tmpnam|:\verb|tmpnam|でつくられる一時ファイル名の最大長
\item \verb|TMP_MAX|:\verb|tmpnam|で作ることができる一時ファイルの最大個数
\begin{multicols}{2}
\item \verb|NULL|:\verb|NULL|ポインタ
\item \verb|SEEK_CUR|:現在のファイル位置指示
\item \verb|SEEK_END|:ファイルの最後
\item \verb|SEEK_SET|:ファイルの先頭
\item \verb|stdin|:標準入力ストリーム
\item \verb|stdout|:標準出力ストリーム
\end{multicols}
\item \verb|stderr|:標準エラー出力ストリーム
\item \verb|int remove(const char *fn)|:\verb|fn|の示すファイルを削除する。
\item \verb|int rename(const char *old,const char *new)|:\verb|old|で示すファイル名を\verb|new|で示すファイル名に変更する。
\item \verb|FILE *tmpfile(void)|:一時ファイルを作成してそのファイルストリームを返す。
\item \verb|char *tmpnam(char *s)|:一意な一時ファイル名を生成して\verb|s|に格納し、その名前を関数値として返す。
\item \verb|int fclose(stream)|:引数のストリームをクローズする。
\item \verb|int fflush(stream)|:引数のストリームをフラッシュする。
\item \verb|FILE *fopen(const char *fn,const char *mode)|:\verb|mode|に従って\verb|fn|の示すファイルをオープンしてストリームを返す。
\item \verb|FILE *freopen(const char *fn,const char *mode,stream)|:\verb|mode|に従って\verb|fn|の示すファイルを再オープンして\verb|stream|にストリームを返す。
\item \verb|void setbuf(stream,char *buf)|:\verb|stream|のバッファリングをバッファ\verb|buf|を用いて行う。
\item \verb|int setvbuf(stream,char *buf,int mode,size_t size)|:\verb|stream|のバッ\\ファリングをバッファ\verb|buf|を用いて\verb|mode,size|に従って行う。
\item \verb|int fprintf(stream,const char *format,...)|:\verb|format|に従い\verb|stream|にデータを出力する。
\item \verb|int fscanf(stream,const char *format,...)|:\verb|format|に従い\verb|stream|からデータを入力する。
\item \verb|int printf(const char *format,...)|:\verb|format|に従い\verb|stdout|にデータを出力する。
\item \verb|int scanf(const char *format,...)|:\verb|format|に従い\verb|stdin|からデータを入力する。
\item \verb|int sprintf(char *s,const char *format,...)|:\verb|format|に従ってデータを文字列\verb|s|に出力する。
\item \verb|int sscanf(char *s,const char *format,...)|:\verb|format|に従ってデータを文字列\verb|s|から入力する。
\item \verb|int vfprintf(stream,const char *format,va_list arg)|:\verb|format|に従い\\ \verb|stream|に可変引数リスト\verb|arg|の内容を出力する。
\item \verb|int vprintf(const char *format,va_list arg)|:\verb|format|に従い\verb|stdout|に可変引数リスト\verb|arg|の内容を出力する。
\item \verb|int vsprintf(char *s,const char *format,va_list arg)|:\verb|format|に従い文字列\verb|s|に可変引数リスト\verb|arg|の内容を出力する。
\item \verb|int fgetc(stream)|:\verb|stream|より1文字入力してその文字を返す。
\item \verb|char *fgets(char *s,int n,stream)|:\verb|stream|より\verb|n|文字入力して\verb|s|に格納し、\verb|s|へのポインタを返す。
\item \verb|int fputc(int c,stream)|:\verb|c|を\verb|stream|に出力する。
\item \verb|int fputs(const char *s,stream)|:\verb|s|を\verb|stream|に出力する。
\item \verb|int getc(stream)|:\verb|stream|より1文字入力してその文字を返す。
\item \verb|int getchar(void)|:\verb|stdin|より1文字入力してその文字を返す。
\item \verb|char *gets(char *s)|:\verb|stdin|より1行入力して\verb|s|に格納し、\verb|s|へのポインタを返す(改行コードが\verb|NULL|文字に置き換えられる)。
\item \verb|int putc(int c,stream)|:\verb|c|を\verb|stream|に出力する。
\item \verb|int putchar(int int c)|:\verb|stdout|に\verb|c|を出力して\verb|c|を返す。
\item \verb|int puts(const char *s)|:\verb|s|を\verb|stdout|に出力する。(\verb|NULL|文字が抜かれて改行コードが補われる。)
\item \verb|int ungetc(int c,stream)|:\verb|stream|に\verb|c|を戻し、\verb|c|を返す。
\item \verb|size_t fread(void *ptr,size_t size,size_t nmemb,stream)|:\\ \verb|stream|から\verb|size|分のデータ\verb|nmemb|個を入力して\verb|ptr|にセットする。
\item \verb|size_t fwrite(const void *ptr,size_t size,size_t nmemb,stream)|:\\ \verb|stream|に\verb|size|分のデータ\verb|nmemb|個を\verb|ptr|から出力する。
\item \verb|int fgetpos(stream,fpos_t *pos)|:\verb|stream|のファイル位置指示子の値を求めて\verb|pos|にセットする。
\item \verb|int fseek(stream,long int offset,int whence)|:\verb|stream|に対しファイル位置\\ \verb|whence|と移動量\verb|offset|を変更する。
\item \verb|int fsetpos(stream,fpos_t *pos)|:\verb|stream|のファイル位置を\verb|pos|に変更する。
\item \verb|long int ftell(stream)|:引数のファイルポジションインジケータの値を返す。
\item \verb|void rewind(stream)|:引数のファイルポジションインジケータをファイル先頭にセットする。
\item \verb|void clearerr(stream)|:引数の\verb|EOF|及びエラー指示子をクリアする。
\item \verb|int feof(stream)|:引数のファイル終端指示子を調べ、\verb|EOF|ならば非0を返す。
\item \verb|int ferror(stream)|:引数のエラー状態指示子を調べ、エラーならば非0を返す。
\item \verb|void perror(const char *s)|:\verb|errno|に対応するエラーメッセージを\verb|stderr|に出力する。
\end{itemize}
以下はC99において追加された関数である。
\begin{itemize}
\item \verb|int snprintf(char *s,size_t n,const char *format,...)|:\verb|format|に従ってデータを\verb|n|文字分、文字列\verb|s|に出力する。返却値は出力文字数。
\item \verb|int vsnprintf(char *s,size_t n,const char *format,va_list arg)|\\:\verb|format|に従い文字列\verb|s|に\verb|n|文字分可変引数リスト\verb|arg|の内容を出力する。返却値は出力文字数。
\item \verb|int vscanf(const char *format,va_list arg)|:\verb|scanf|の可変個数引数部分を可変引数リスト\verb|va_list|に変更したもの。
\item \verb|int vsscanf(const char *s,const char *format,va_list arg)|:\verb|sscanf|の可変個数引数部分を\verb|va_list|に変更したもの。
\item \verb|int vfscanf(stream,const char *format,va_list arg)|:\verb|fscanf|の可変個数引数部分を\verb|va_list|に変更したもの。
\end{itemize}

\section{stdlib.h(ユーティリティ)}
以下特に断らない限り、\verb|n|及び\verb|size|は\verb|size_t|型とし、\verb|str|は\verb|const char *|型とする。
\begin{itemize}
\item \verb|div_t|:\verb|div()|の返す型
\item \verb|ldiv_t|:\verb|ldiv()|の返す型
\item \verb|size_t|:\verb|sizeof|演算子の返す符号なし整数型
\item \verb|wchar_t|:\verb|wide|文字の型
\item \verb|EXIT_FAILURE|:プログラムの実行が失敗したことを示す
\item \verb|EXIT_SUCCESS|:プログラムの実行が成功したことを示す
\item \verb|MB_CUR_MAX|:その時点でのマルチバイト文字を表現するのに必要な最大バイト数
\item \verb|NULL|:\verb|NULL|ポインタ
\item \verb|RAND_MAX|:疑似乱数の最大値
\item \verb|double atof(str)|:\verb|str|を浮動小数点数に変換して返す。
\item \verb|int atoi(str)|:\verb|str|を整数に変換して返す。
\item \verb|long int atol(str)|:\verb|str|を\verb|long|型整数にして返す。
\item \verb|double strtod(str,char **endptr)|:\verb|str|を浮動小数点数に変換し(先頭の空白文字はスキップされる)、浮動小数点より後ろの文字列へのポインタを\verb|*endptr|に格納する。返却値は変換された浮動小数点数。
\item \verb|long int strtol(str,char **endptr,int base)|:\verb|str|を\verb|base|によって指定された記数法にしたがって\verb|long|型に変換し(先頭の空白文字はスキップされる)、その数値より後ろの文字列へのポインタを\verb|*endptr|に格納する。返却値は変換された数。
\item \verb|unsigned long int strtoul(str,char **endptr,int base)|:\verb|str|を\verb|base|によって指定された記数法にしたがって\verb|unsigned long|型に変換し(先頭の空白文字はスキップされる)、その数値より後ろの文字列へのポインタを\verb|*endptr|に格納する。返却値は変換された数。
\item \verb|int rand(void)|:0から\verb|RAND_MAX|の範囲で疑似乱数を発生させ、その値を返す。
\item \verb|void srand(unsigned int seed)|:引数を疑似乱数発生ルーチンの種として与える。
\item \verb|void *calloc(n,size)|:\verb|size|バイト\verb|n|個分の動的メモリを割り当て、それを0クリアする。
\item \verb|void free(voind *ptr)|:引数のポインタで与えられた動的メモリ領域を解放する。
\item \verb|void *malloc(size)|:\verb|size|バイト分の動的メモリを割り当てる。
\item \verb|void *realloc(void *ptr,size)|:\verb|ptr|で示される動的メモリを\verb|size|バイト分の大きさで再割り当てする。
\item \verb|void abort(void)|:プログラムを異常終了する。
\item \verb|int atexit(void (*func)(void))|:プログラム終了時に実行する関数を登録する。
\item \verb|void exit(int status)|:引数を返却値としてプログラムを終了する。
\item \verb|char *getenv(str)|:環境変数\verb|str|に対応する文字列へのポインタを返す。
\item \verb|int system(str)|:\verb|str|に示されるプログラム(コマンドライン形式)を実行する。
\item \begin{verbatim}
void *bsearch(const void *key,const void *base,n,size,
  int (*compare)(const void *, const void *))
\end{verbatim}:配列(\verb|base|、要素毎のサイズ\verb|size|、要素数\verb|n|)内のデータ中のキー\verb|key|に一致するデータをバイナリサーチで検索し、その要素へのポインタを返す。比較は\verb|compare|で示される比較関数によって行う。
\item \begin{verbatim}
void qsort(void *base,n,size,
  int (*compare)(const void *, const void *))
\end{verbatim}:配列(\verb|base|、要素毎のサイズ\verb|size|、要素数\verb|n|)を\verb|compare|で示される比較関数に従ってソートする。
\item \verb|int abs(int j)|:整数引数の絶対値を返す。
\item \verb|long int labs(long int j)|:\verb|long|型引数の絶対値を返す。
\item \verb|div_t div(int num,int denom)|:\verb|num|/\verb|denom|の商と剰余を\verb|div_t|型で返す。
\item \verb|ldiv_t ldiv(long int num,long int denom)|:\verb|num|/\verb|denom|の\\商と剰余を\verb|ldiv_t|型で返す。
\item \verb|int mblen(str,n)|:マルチバイト文字列\verb|str|を最大\verb|n|バイトまで検査して、次のマルチバイト文字のバイト数を返す。
\item \verb|int mbtowc(wchar_t *pwc,str,n)|:マルチバイト文字\verb|str|を最初から\verb|n|バイトまで\verb|wide|文字(\verb|pwc|)に変換する。
\item \verb|int wctomb(char *s,wchar_t wchar)|:\verb|wide|文字(\verb|wchar|)をマルチバイト文字\verb|s|に変換する。
\item \verb|size_t mbstowcs(wchar_t *pwcs,str,n)|:マルチバイト文字列\verb|str|を最初から\verb|n|バイトまで\verb|wide|文字列(\verb|pwcs|)に変換する。
\item \verb|size_t wcstombs(char *s,const wchar_t *pwcs,n)|:\verb|wide|文字列(\verb|pwcs|)を最初から\verb|n|バイトまでマルチバイト文字列\verb|s|に変換する。
\end{itemize}
以下はC99において追加された型/関数である。
\begin{itemize}
\item \verb|lldiv_t|:\verb|lldiv()|の返す型
\item \verb|void _Exit(int status)|:プログラムを正常終了する。
\item \verb|float strtof(str,char **endp)|:\verb|strtod|の\verb|float|版
\item \verb|long double strtold(str,char **endp)|:\verb|strtod|の\verb|long double|版
\item \verb|long long int strtoll(str,char **endptr,int base)|:\verb|strtol|の\\ \verb|long long int|版
\item \verb|unsigned long long int strtoull(str,char **endptr,int base)|:\verb|strtol|の\\ \verb|unsigned long long int|版
\item \verb|long long int atoll(str)|:\verb|atoi|の\verb|long long int|版
\item \verb|long long int llabs(long long int j)|:\verb|abs|の\verb|long long int|版
\item \verb|lldiv_t lldiv(long long int num,long long int denom)|:\verb|ldiv|の\\ \verb|long long int|版
\end{itemize}

\section{string.h(文字列操作)}
以下特に断らない限り、\verb|s1|,\verb|s2|,\verb|s|はいずれも\verb|char *|型とし、\verb|m1|,\verb|m2|,\verb|m|は\verb|void *|型とする。但し、\verb|const s2|等と書いた場合は、\verb|const char *|型等、前に\verb|const|を付すものとする。また、\verb|n|は\verb|size_t|型とする。
\begin{itemize}
\item \verb|size_t|:\verb|sizeof|演算子の返す符号なし整数型
\item \verb|NULL|:\verb|NULL|ポインタ
\item \verb|void *memcpy(m1,const m2,n)|:\verb|m2|を\verb|m1|に\verb|n|バイト分コピーする。
\item \verb|void *memmove(m1,const m2,n)|:\verb|m2|を\verb|m1|に\verb|n|バイト分コピーする。(\verb|m2|と\verb|m1|が重なっても良い。)
\item \verb|int memcmp(const m1,const m2,n)|:\verb|m1|,\verb|m2|を\verb|n|バイトまで比較し、一致したら0,一致しなければ非0を返す。
\item \verb|void *memchr(const m,int c,n)|:文字\verb|c|を\verb|m|の最初の\verb|n|バイト中から探し、あればその文字へのポインタを、なければ\verb|NULL|を返す。
\item \verb|void *memset(m,int c,n)|:\verb|m|を文字\verb|c|で\verb|n|バイト分埋める。
\item \verb|char *strcpy(s1,const s2)|:\verb|s1|に\verb|s2|をコピーする。
\item \verb|char *strncpy(s1,const s2,n)|:\verb|s1|に\verb|s2|を最大\verb|n|バイトまでコピーする。
\item \verb|char *strcat(s1,const s2)|:\verb|s1|の後ろに\verb|s2|を連結する。
\item \verb|char *strncat(s1,const s2,n)|:\verb|s1|の後ろに\verb|s2|を最大\verb|n|バイトまで連結する。
\item \verb|int strcmp(const s1,const s2)|:\verb|s1|,\verb|s2|を比較し、一致したら0,一致しなければ非0を返す。
\item \verb|int strcoll(const s1,const s2)|:地域情報を使い\verb|s1|,\verb|s2|を比較し、一致したら0,一致しなければ非0を返す。
\item \verb|int strncmp(const s1,const s2,n)|:\verb|s1|,\verb|s2|を\verb|n|バイトまで比較し、一致したら0,一致しなければ非0を返す。
\item \verb|size_t strxfrm(s1,const s2,n)|:\verb|s2|を地域情報にしたがって\verb|s1|に最初の\verb|n|バイトだけ変換する。
\item \verb|char *strchr(const s,int c)|:文字\verb|c|を\verb|s|中から探し、あればその文字へのポインタを、なければ\verb|NULL|を返す。
\item \verb|size_t strcspn(const s1,const s2)|:\verb|s2|に含まれない文字だけで構成される文字列を\verb|s1|から探し、その最初の部分の長さを返す。
\item \verb|char *strpbrk(const s1,const s2)|:\verb|s2|中の文字が\verb|s1|に出てくる、その最初の文字へのポインタを返す。
\item \verb|char *strrchr(const s,int c)|:\verb|s|中で文字\verb|c|が現れる最後の位置のポインタを返す。
\item \verb|size_t strspn(const s1,const s2)|:\verb|s2|に含まれる文字だけで構成される文字列を\verb|s1|から探し、その最初の部分の長さを返す。
\item \verb|char *strstr(s)|:\verb|s2|が\verb|s1|に出てくる最初の文字へのポインタを返す。
\item \verb|char *strtok(s1,const s2)|:\verb|s1|を区切り記号文字列\verb|s2|にしたがってトークンに分割する。\verb|s1|は2回目以降の呼び出しにおいて\verb|NULL|を指定する。トークンがあればそのポインタを,なければ\verb|NULL|を返す。
\item \verb|char *strerror(int errnum)|:エラー番号\verb|errnum|を文字列に変換し、そのポインタを返す。
\item \verb|size_t strlen(const s)|:\verb|s|の長さを返す。
\end{itemize}

\section{time.h(時間)}
\begin{itemize}
\item \verb|clock_t|:\verb|clock()|の返却値の型
\item \verb|time_t|:カレンダ時間の型
\item \verb|size_t|:\verb|sizeof|演算子の返す符号なし整数型
\item \verb|CLOCKS_PER_SEC|:\verb|clock_t|における1秒間の数
\item \verb|NULL|:\verb|NULL|ポインタ
\item \verb|clock_t clock(void)|プログラムの実行に要した経過時間を返す。
\item \verb|double difftime(time_t time1,time_t time2)|:\verb|time1|と\verb|time2|の差を秒で返す。
\item \verb|time_t mktime(struct tm *timeptr)|:ローカル時間\verb|timeptr|をカレンダ時間に変換して返す。
\item \verb|time_t time(time_t *timer)|:現在のカレンダ時間を\verb|timer|にセットし、カレンダ時間を返す。
\item \verb|char *asctime(const struct tm *timeptr)|:ローカル時間\verb|timeptr|を文字列に変換して返す。
\item \verb|char *ctime(const time_t *timer)|:カレンダ時間\verb|timer|を文字列に変換して返す。
\item \verb|struct tm *gmtime(const time_t *timer)|:カレンダ時間\verb|timer|を世界標準時に変換して返す。
\item \verb|struct tm *localtime(const time_t *timer)|:カレンダ時間\verb|timer|をローカル時間に変換して返す。
\item \begin{verbatim}
size_t strftime(char *s,size_t max,
  const char *fm,const struct tm *tptr)
\end{verbatim}:ローカル時間\verb|tptr|を表示形式\verb|fm|に従い、最大\verb|max|文字まで変換し、\verb|s|にセットする。
\end{itemize}

\section{iso646.h(代替綴・C95)}
本ヘッダはC95において追加されたヘッダである。演算子をマクロを用いて記すためのヘッダであり、以下はいずれもマクロである。
\begin{multicols}{2}
\begin{itemize}
\item \verb|and|:置き換える演算子は\verb|&&|
\item \verb|and_eq|:置き換える演算子は\verb|&=|
\item \verb|bitand|:置き換える演算子は\verb|&|
\item \verb|bitor|:置き換える演算子は$|$
\item \verb|compl|:置き換える演算子は\verb|~|
\item \verb|not|:置き換える演算子は\verb|!|
\item \verb|not_eq|:置き換える演算子は\verb|!=|
\item \verb|or|:置き換える演算子は$||$
\item \verb|or_eq|:置き換える演算子は$|=$
\item \verb|xor|:置き換える演算子は\verb|^|
\item \verb|xor_eq|:置き換える演算子は\verb|^=|
\end{itemize}
\end{multicols}

\section{wchar.h(ワイド文字・C95)}
本ヘッダはC95において追加されたヘッダである。ひとまず、型とマクロを記す。
\begin{itemize}
\item \verb|wchar_t|:\verb|wide|文字の型
\item \verb|wint_t|:\verb|wchar_t|に加えて、拡張文字で表示されない値をひとつ以上示す広義整数型
\item \verb|mbstate_t|:マルチバイト文字からワイド文字への変換状態を示す型
\item \verb|size_t|:\verb|sizeof|演算子の返す符号なし整数型
\item \verb|NULL|:\verb|NULL|ポインタ
\item \verb|WCHAR_MIN|:\verb|wchar_t|型の最小値
\item \verb|WCHAR_MAX|:\verb|wchar_t|型の最大値
\item \verb|WEOF|:ファイル終端を示す\verb|wint_t|型の値
\end{itemize}
次いで、他のヘッダの関数と直接には対応しない関数を示す。\verb|c_(型)|は\verb|const (型)|を示す。
\begin{itemize}
\item \verb|fwide(FILE *stream,int mode)|:\verb|stream|の入出力単位を\verb|mode|が負の場合はバイト単位、正の場合はワイド文字単位に設定する。0の場合は設定を変更しない。\verb|mode|と同符号の値を返却する。
\item \verb|wint_t btowc(int c)|:引数の1バイト文字をワイド文字に変換して返す。
\item \verb|int wctob(wint_t c)|:引数のワイド文字を1バイト文字に変換して返す。
\item \verb|int mbsinit(c_mbstate_t *ps)|:引数の\verb|mbstate_t|オブジェクトが初期変換状態を表すかどうかを判定し、表す場合は非0、表さない場合は0を返す。
\item \verb|size_t mbrlen(c_char *s,size_t n,mbstate_t *ps)|:マルチバイト文字のバイト長を取得する。
\item \verb|size_t mbrtowc(wchar_t *c,c_char *s, size_t n,mbstate_t *ps)|:変換状態格納領域を\verb|ps|とし、マルチバイト文字をワイド文字に変換する。
\item \verb|size_t wcrtomb(char *s,wchar_t wc,mbstate_t *ps)|:変換状態格納領域を\verb|ps|とし、ワイド文字をマルチバイト文字に変換する。
\item \verb|size_t mbsrtowcs(wchar_t *p,c_char **s, size_t n,mbstate_t *ps)|:変換状態格納領域を\verb|ps|とし、\verb|s|の示すマルチバイト文字列を\verb|n|バイト分ワイド文字列に変換して\verb|p|に格納する。返却値はエラーが出れば-1、処理成功時は変換に成功したヌル文字以外の文字の数である。
\item \verb|size_t wcsrtombs(char *s,c_wchar_t **src,size_t n,mbstate_t *ps)|:変換状態格納領域を\verb|ps|とし、\verb|src|の示すワイド文字列を\verb|n|バイト分マルチバイト文字列に変換して\verb|s|に格納する。返却値はエラーが出れば-1、処理成功時は変換に成功したヌル文字以外の文字の数である。
\end{itemize}
以下に示す関数は\verb|stdio.h|,\verb|stdlib.h|等の関数の\verb|char|の部分を\verb|wchar_t|に変更したものである。それ故、型や引数は省き、対応する関数を挙げておく。
\begin{multicols}{2}
\begin{itemize}
\item \verb|fwprintf|:\verb|fprintf|に対応
\item \verb|fwscanf|:\verb|fscanf|に対応
\item \verb|swprintf|:\verb|snprintf|に対応
\item \verb|swscanf|:\verb|sscanf|に対応
\item \verb|vfwprintf|:\verb|vfprintf|に対応
\item \verb|vswprintf|:\verb|vsprintf|に対応
\item \verb|vwprintf|:\verb|vprintf|に対応
\item \verb|wprintf|:\verb|printf|に対応
\item \verb|wscanf|:\verb|scanf|に対応
\item \verb|fgetwc|:\verb|fgetc|に対応
\item \verb|fgetws|:\verb|fgets|に対応
\item \verb|fputwc|:\verb|fputc|に対応
\item \verb|fputws|:\verb|fputs|に対応
\item \verb|getwc|:\verb|getc|に対応
\item \verb|getwchar|:\verb|getchar|に対応
\item \verb|putwc|:\verb|putc|に対応
\item \verb|putwchar|:\verb|putchar|に対応
\item \verb|ungetwc|:\verb|ungetc|に対応
\item \verb|wcstod|:\verb|strtod|に対応
\item \verb|wcstol|:\verb|strtol|に対応
\item \verb|wcstoul|:\verb|strtoul|に対応
\item \verb|wcscpy|:\verb|strcpy|に対応
\item \verb|wcsncpy|:\verb|strncpy|に対応
\item \verb|wmemcpy|:\verb|memcpy|に対応
\item \verb|wmemmove|:\verb|memmove|に対応
\item \verb|wcscat|:\verb|strcat|に対応
\item \verb|wcsncat|:\verb|strncat|に対応
\item \verb|wcscmp|:\verb|strcmp|に対応
\item \verb|wcscoll|:\verb|strcoll|に対応
\item \verb|wcsncmp|:\verb|strncmp|に対応
\item \verb|wcsxfrm|:\verb|strxfrm|に対応
\item \verb|wmemcmp|:\verb|memcmp|に対応
\item \verb|wcschr|:\verb|strchr|に対応
\item \verb|wcscspn|:\verb|strcspn|に対応
\item \verb|wcspbrk|:\verb|strpbrk|に対応
\item \verb|wcsrchr|:\verb|strrchr|に対応
\item \verb|wcsspn|:\verb|strspn|に対応
\item \verb|wcsstr|:\verb|strstr|に対応
\item \verb|wcstok|:\verb|strtoken|に対応
\item \verb|wmemchr|:\verb|memchr|に対応
\item \verb|wcslen|:\verb|strlen|に対応
\item \verb|wmemset|:\verb|memset|に対応
\item \verb|wcsftime|:\verb|strftime|に対応
\end{itemize}
\end{multicols}
以下は、C99において追加された関数である。
\begin{multicols}{2}
\begin{itemize}
\item \verb|wcstoull|:\verb|strtoull|に対応
\item \verb|wcstoll|:\verb|strtoll|に対応
\item \verb|wcstof|:\verb|strtof|に対応
\item \verb|wcstold|:\verb|strtold|に対応
\item \verb|vwscanf|:\verb|vscanf|に対応
\item \verb|vswscanf|:\verb|vsscanf|に対応
\item \verb|vfwscanf|:\verb|vfscanf|に対応
\end{itemize}
\end{multicols}

\section{wctype.h(ワイド文字変換・C95)}
本ヘッダはC95において追加されたヘッダである。まず、型とマクロを示す。
\begin{itemize}
\item \verb|wctype_t|:ワイド文字の種別を表す型
\item \verb|wctrans_t|:あるワイド文字を他のワイド文字に変換できるマッピングを表現する型
\item \verb|wint_t|:\verb|wchar_t|に加えて、拡張文字で表示されない値をひとつ以上示す広義整数型
\item \verb|WEOF|:ファイル終端を示す\verb|wint_t|型の値
\end{itemize}
他のヘッダの関数と対応のない関数は以下の通り。
\begin{itemize}
\item \verb|int iswctype(wint_t wc,wctype_t desc)|:\verb|wc|が\verb|desc|に属するワイド文字か否か判定し、属する場合は非0を、属さない場合は0を返す。
\item \verb|wint_t towctrans(wint_t wc,wctrans_t desc)|:\verb|desc|の変換に従って\verb|wc|を変換して返却する。
\item \verb|wctrans_t wctrans(const char *p)|:\verb|p|によって識別される変換の値を返却する。
\item \verb|wctype_t wctype(const char *p)|:\verb|p|によって識別されるワイド文字の種別を返却する。
\end{itemize}
上記以外は対応する関数が\verb|ctype.h|にある。対応する関数の引数の型を\verb|wint_t|に変更したものが本ヘッダ内の関数である。
\begin{multicols}{2}
\begin{itemize}
\item \verb|iswalnum|:\verb|isalnum|に対応
\item \verb|iswalpha|:\verb|isalpha|に対応
\item \verb|iswcntrl|:\verb|iscntrl|に対応
\item \verb|iswdigit|:\verb|isdigit|に対応
\item \verb|iswgraph|:\verb|isgraph|に対応
\item \verb|iswlower|:\verb|islower|に対応
\item \verb|iswprint|:\verb|isprint|に対応
\item \verb|iswpunct|:\verb|ispunct|に対応
\item \verb|iswspace|:\verb|isspace|に対応
\item \verb|iswupper|:\verb|isupper|に対応
\item \verb|iswxdigit|:\verb|isxdigit|に対応
\item \verb|towlower|:\verb|tolower|に対応
\item \verb|towupper|:\verb|toupper|に対応
\end{itemize}
\end{multicols}
以下は、C99において追加された関数である。
\begin{itemize}
\item \verb|iswblank|:\verb|isblank|に対応
\end{itemize}

\section{complex.h(複素数演算・C99)}
本ヘッダはC99において追加されたヘッダである。まず、マクロを掲載しておく。
\begin{itemize}
\item \verb|complex|:型名\verb|_Complex|を表す
\item \verb|_Complex_I|:\verb|const float _Complex|型の虚数単位
\item \verb|imaginary|:型名\verb|_Imaginary|を表す
\item \verb|_Imaginary_I|:\verb|const float _Imaginary|型の虚数単位
\item \verb|I|:虚数単位(型は環境依存)
\end{itemize}
本ヘッダ内の関数は\verb|math.h|に対応した関数のあるものが多い。以下にそれを挙げておく。なお、関数の型・引数の型とも、特記しないものは\verb|double complex|型とする。
\begin{multicols}{2}
\begin{itemize}
\item \verb|cacos(z)|:\verb|acos|に対応
\item \verb|casin(z)|:\verb|asin|に対応
\item \verb|catan(z)|:\verb|atan|に対応
\item \verb|ccos(z)|:\verb|cos|に対応
\item \verb|csin(z)|:\verb|sin|に対応
\item \verb|ctan(z)|:\verb|tan|に対応
\item \verb|ccosh(z)|:\verb|cosh|に対応
\item \verb|csinh(z)|:\verb|sinh|に対応
\item \verb|ctanh(z)|:\verb|tanh|に対応
\item \verb|cacosh(z)|:\verb|acosh|に対応
\item \verb|casinh(z)|:\verb|asinh|に対応
\item \verb|catanh(z)|:\verb|atanh|に対応
\item \verb|cexp(z)|:\verb|exp|に対応
\item \verb|clog(z)|:\verb|log|に対応
\item \verb|cpow(x,y)|:\verb|pow|に対応
\item \verb|csqrt(z)|:\verb|sqrt|に対応
\item \verb|double cabs(z)|:\verb|fabs|に対応
\end{itemize}
\end{multicols}
以下は複素数特有の関数である。
\begin{multicols}{2}
\begin{itemize}
\item \verb|conj(z)|:共役複素数を返す。$\overline{z}$
\item \verb|cproj(z)|:リーマン球面上への射影値を返す。
\item \verb|double cart(z)|:仰角を返す。
\item \verb|double creal(z)|:実部を返す。
\item \verb|double cimag(z)|:虚部を返す。
\end{itemize}
\end{multicols}
本ヘッダも\verb|math.h|同様、上記に挙げた関数の関数名終端に\verb|f|を付すと関数・引数とも\verb|float complex|型に、\verb|l|を付すと\verb|long double complex|型になる(\verb|double|以外で特記しているものを除く)。また、\verb|math.h|も\verb|complex.h|も含め、\verb|tgmath.h|をインクルードすることで\verb|math.h|の\verb|double|型関数名(複素数に特有の関数は\verb|complex.h|の\verb|double complex|型関数名)で関数が使えるようになる。

\section{fenv.h(浮動小数点数環境・C99)}
本ヘッダはC99において追加されたヘッダである。
\begin{itemize}
\item \verb|fenv_t|:現在の浮動小数点数環境情報を格納する型
\item \verb|fexcept|:例外に関する情報を格納する型
\begin{multicols}{2}
\item \verb|FE_DIVBYZERO|:ゼロ除算例外
\item \verb|FE_INEXACT|:不正確例外
\item \verb|FE_INVALID|:不正操作例外
\item \verb|FE_OVERFLOW|:オーバーフロー例外
\item \verb|FE_UNDERFLOW|:アンダーフロー例外
\item \verb|FE_ALL_EXCEPT|:処理系定義の全例外
\item \verb|FE_DOWNWARD|:$-\infty$の方向へ丸める
\item \verb|FE_TONEAREST|:最も近い値へ丸める
\item \verb|FE_TOWARDZERO|:0方向へ丸める
\item \verb|FE_UPWARD|:$+\infty$の方向へ丸める
\item \verb|FE_DFL_ENV|:デフォルトの浮動小数点数環境
\end{multicols}
\item \verb|int fegetexceptflag(fexcept_t *f, int e)|:\verb|e|で指定された例外フラグを実装定義依存の値として\verb|f|に格納する。もし格納に成功した場合には0を、失敗した場合には、非0を返す。
\item \verb|int fesetexceptflag(const fexcept_t *f, int e)|:\verb|f|で指定されたオブジェクトでの表現に従い、\verb|e|で指定された例外フラグに対する完全な状態をセットする。\verb|e|が0か状態の設定に成功した場合には0を返す。それ以外は非0を返す。この関数ではフラグの状態をセットするだけで例外は発生しない。
\item \verb|int feclearexcept(int e)|:\verb|e|で指定された例外フラグのクリアを試みる。\verb|e|が0かクリアが成功した場合に0を返す。それ以外は非0を返す。
\item \verb|int fetestexcept(int e)|:\verb|e|で指定された例外フラグが現在セットされているか検査し、セットされている例外フラグ値を返す。
\item \verb|int feraiseexcept(int e)|:\verb|e|で指定された例外を発生させる。\verb|e|が0か例外の発生に成功した場合は0を返す。それ以外は非0を返す。
\item \verb|int fegetround(void)|:現在の丸めモードを丸めフラグの値で返す。 
\item \verb|int fesetround(int r)|:\verb|r|で指定された丸めモードにセットする。引数が丸めフラグの値でないときは状態は変更されない。引数で指定された丸め方向に設定できたときに限り0を返す。 
\item \verb|int fegetenv(fenv_t *e)|:現在の浮動小数点数環境を\verb|e|に格納する。成功すると0を返す。 
\item \verb|int fesetenv(const fenv_t *e)|:現在の浮動小数点数環境に指定された浮動小数点数環境\verb|e|を設定する。設定に成功すると0を返す。 
\item \verb|int feholdexcept(fenv_t *e)|:現在の浮動小数点数環境を\verb|e|に格納し、例外フラグをクリアし、可能であれば全例外に対し(例外でも継続する)非停止モードに設定する。もし非停止モードに設定できたときは0を返す。 
\item \verb|int feupdateenv(const fenv_t *e)|:現在発生した例外を一時領域に格納し、\verb|e|で指定された浮動小数点数環境を設定した後、一時領域に格納した例外を発生させる。例外発生に成功した場合には0を返す。 
\end{itemize}

\section{inttypes.h(整数型書式・C99)}
本ヘッダはC99において追加されたヘッダである。書式設定用マクロについては本文参照。ここでは対応関数を示すに留める。
\begin{multicols}{2}
\begin{itemize}
\item \verb|imaxabs|:\verb|abs|の\verb|intmax_t|版
\item \verb|imaxdiv|:\verb|div|の\verb|intmax_t|版
\item \verb|strtoimax|:\verb|strtol|の\verb|intmax_t|版
\item \verb|strtoumax|:\verb|strtol|の\verb|uintmax_t|版
\item \verb|wcstoimax|:\verb|wcstol|の\verb|intmax_t|版
\item \verb|wcstoumax|:\verb|wcstol|の\verb|uintmax_t|版
\end{itemize}
\end{multicols}

\section{stdbool.h(論理・C99)}
本ヘッダはC99において追加されたヘッダである。マクロのみのヘッダである。
\begin{itemize}
\item \verb|bool|:\verb|_Bool|型を示す
\item \verb|true|:整数定数1に展開する
\item \verb|false|:整数定数0に展開する
\item \verb|__bool_true_false_are_defined|:整数定数1に展開する
\end{itemize}

\section{stdint.h(整数型管理・C99)}
本ヘッダはC99において追加されたヘッダである。マクロのみのヘッダである。なお、マクロ中の\verb|N|には通常、8,16,32,64の何れかが入り、これが幅指定となる。
\begin{itemize}
\item \verb|INTN_MIN|:幅指定符号付き整数型の最小値
\item \verb|INTN_MAX|:幅指定符号付き整数型の最大値
\item \verb|UINTN_MAX|:幅指定符号無し整数型の最大値
\item \verb|INT_LEASTN_MIN|:最小幅指定符号付き整数型の最小値
\item \verb|INT_LEASTN_MAX|:最小幅指定符号付き整数型の最大値
\item \verb|UINT_LEASTN_MAX|:最小幅指定符号無し整数型の最大値
\item \verb|INT_FASTN_MIN|:最速最小幅指定符号付き整数型の最小値
\item \verb|INT_FASTN_MAX|:最速最小幅指定符号付き整数型の最大値
\item \verb|UINT_FASTN_MAX|:最速最小幅指定符号無し整数型の最大値
\item \verb|INTPTR_MIN|:ポインタ保持可能な符号付き整数型の最小値
\item \verb|INTPTR_MAX|:ポインタ保持可能な符号付き整数型の最大値
\item \verb|UINTPTR_MAX|:ポインタ保持可能な符号付き整数型の最大値
\item \verb|INTMAX_MIN|:最大幅符号付き整数型の最小値
\item \verb|INTMAX_MAX|:最大幅符号付き整数型の最大値
\item \verb|UINTMAX_MAX|:最大幅符号無し整数型の最大値
\item \verb|PTRDIFF_MIN|:\verb|ptrdiff_t|の限界値下限
\item \verb|PTRDIFF_MAX|:\verb|ptrdiff_t|の限界値上限
\item \verb|SIG_ATOMIC_MIN|:\verb|sig_atomic_t|の限界値下限
\item \verb|SIG_ATOMIC_MAX|:\verb|sig_atomic_t|の限界値上限
\item \verb|SIZE_MAX|:\verb|size_t|の限界値
\item \verb|WCHAR_MIN|:\verb|wchar_t|の限界値下限
\item \verb|WCHAR_MAX|:\verb|wchar_t|の限界値上限
\item \verb|WINT_MIN|:\verb|wint_t|の限界値下限
\item \verb|WINT_MAX|:\verb|wint_t|の限界値上限
\item \verb|INTN_C(値)|:値を\verb|int_leastN_t|に対応する整数定数式に展開
\item \verb|UINTN_C(値)|:値を\verb|uint_leastN_t|に対応する整数定数式に展開
\item \verb|INTMAX_C(値)|:値を\verb|intmax_t|である整数定数式に展開
\item \verb|UINTMAX_C(値)|:値を\verb|uintmax_t|である整数定数式に展開
\end{itemize}

\section{tgmath.h(型総称数学関数・C99)}
本ヘッダはC99において追加されたヘッダである。このヘッダの中の関数マクロはmath.hあるいはcomplex.hのdoubleの関数と同名である。引数の型によって関数を使い分けるのがこのヘッダの目的である。詳細は本文を参照されたい。

\newpage
